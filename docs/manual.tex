% -*- coding: utf-8 -*-
% Copyright 2014–2015 Justin Lardinois
%
% Licensed under the Apache License, Version 2.0 (the "License");
% you may not use this file except in compliance with the License.
% You may obtain a copy of the License at
%
%   http://www.apache.org/licenses/LICENSE-2.0
%
% Unless required by applicable law or agreed to in writing, software
% distributed under the License is distributed on an "AS IS" BASIS,
% WITHOUT WARRANTIES OR CONDITIONS OF ANY KIND, either express or implied.
% See the License for the specific language governing permissions and
% limitations under the License.
%%%%%%%%%%%%%%%%%%%%%%%%%%%%%%%%%%%%%%%%%%%%%%%%%%%%%%%%%%%%%%%%%%%%%%%%%%%%%%%
% docs/manual.tex - a component of pCRP
% instructions on how to use pCRP

\documentclass[12pt]{article}
\usepackage{fullpage}
\usepackage{hyperref}
\parindent 0pt

\title{pCRP 1.0.0 User Manual}
\author{Justin Lardinois}
\date{}

\begin{document}
\maketitle

\section{Introduction}
	pCRP is a web app for managing conference paper uploads and reviews.
	pCRP is inspired by Eddie Kohler's HotCRP \cite{kohler}, but is easier
	to set up and manage because it runs on Google App Engine, rather than
	your own Linux server, PHP runtime, and MySQL server.

\section{Setup}
	You need Python 2.7.x (I recommend CPython) and the Google App Engine
	SDK for Python \cite{gaesdk} installed on your local machine. You also
	need pip, unless you prefer to install the dependencies by hand. pip
	is included with Python 2.7.9; you'll probably have to install it
	yourself if you have an older version.
	\\\\
	Clone the pcrpdev repository from
	\url{https://bitbucket.org/JustinLardinois/pcrpdev/}, then run
	\texttt{deps.py} in the root of the repository to install the
	necessary libraries.
	\\\\
	Now navigate to the Administration Console at
	\url{http://appengine.google.com/} and click \textbf{Create Application}.
	Under \textbf{Authentication Options (Advanced)}, \textit{make sure}
	to select \textbf{Open to all Google Accounts users (default)}.
	\textbf{Restricted to the following Google Apps domain} works,
	but only makes sense if all of your conference attendees all use
	the same Google Apps domain (which is unlikely).
	\textbf{Open to all users with an OpenID Provider} is untested,
	and will probably break the app; pCRP was not designed to
	support OpenID.
	\\\\
	From here, follow App Engine's documentation on app deployment
	\cite{deploy}. After deploying, your installation of pCRP can
	be accessed at \textit{application-identifier}.appspot.com.

\section{User Roles}
	There are two types of roles in pCRP: administrator, and program
	committee. By default, a user is not an administrator and not a member
	of the program committee.
	\\\\
	The user that creates the pCRP installation will be made an
	administrator by App Engine. Additional administrators can be added
	on the \textbf{Permissions} tab of the App Engine Administration Console.
	Administrators can access the in-app admin panel, which allows them to
	manage program committee roles and change the conference name, deadlines,
	and messages that appear on the home page and hub. Those messages are
	\textbf{not} HTML escaped, so be careful.
	\\\\
	Program committee members can be assigned to review papers, and program
	committee chairs assign reviews.

\section{Paper Authoring}
	Before the registration deadline, users may register papers. This is the
	only stage where the title, author list, and abstract can be edited. Note
	that only the author who started the paper can view and edit it; the
	additional author fields are simply metadata.
	\\\\
	After the registration deadline has passed, and until the submission
	deadline, authors can upload their papers. Currently, pCRP only allows a
	single PDF file to be uploaded.
	\\\\
	After the review deadline has passed, authors can view the reviews of
	their papers.

\section{Review Preferences}
	After the registration deadline has passed, and until the submission
	deadline, program committee members can browse the registered papers
	and indicate their willingness to review each paper on a scale of
	1 to 10, where 1 is least interested and 10 is most interested. This
	information will be displayed to program committee chairs when they
	assign reviews.

\section{Review Questions}
	Before the submission deadline, program committee chairs can edit the
	list of questions to be used for reviews. Currently, pCRP only allows
	chairs to enter questions that have textual answers.

\section{Review Assignment}
	After the submission deadline has passed, and until the registration
	deadline, program committee chairs can assign paper reviews to members
	of the program committee. Currently, pCRP only supports manual
	assignment. Any number of members may be assigned to a given paper.

\section{Conflicts of Interest}
	At any point, users can indicate conflicts of interest with other users.
	Users should be encouraged to do this as soon as possible.
	\\\\
	pCRP considers conflicts of interest to be unilateral; that is, even if
	only one user in a pair of users says they have a conflict, the conflict
	is considered to be bidirectional.
	\\\\
	Program committee members cannot indicate a review preference for papers
	authored by users they have a conflict of interest with, and cannot be
	assigned to review those papers.

\bibliographystyle{plain}
\bibliography{manual}
\end{document}
