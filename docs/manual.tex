% -*- coding: utf-8 -*-
% Copyright 2014–2015 Justin Lardinois
%
% Licensed under the Apache License, Version 2.0 (the "License");
% you may not use this file except in compliance with the License.
% You may obtain a copy of the License at
%
%   http://www.apache.org/licenses/LICENSE-2.0
%
% Unless required by applicable law or agreed to in writing, software
% distributed under the License is distributed on an "AS IS" BASIS,
% WITHOUT WARRANTIES OR CONDITIONS OF ANY KIND, either express or implied.
% See the License for the specific language governing permissions and
% limitations under the License.
%%%%%%%%%%%%%%%%%%%%%%%%%%%%%%%%%%%%%%%%%%%%%%%%%%%%%%%%%%%%%%%%%%%%%%%%%%%%%%%
% docs/manual.tex - a component of pCRP
% instructions on how to use pCRP

\documentclass[12pt]{article}
\usepackage{fullpage}
\usepackage{hyperref}
\parindent 0pt

\title{pCRP User Manual}
\author{Justin Lardinois}
\date{}

\begin{document}
\maketitle

\section{Introduction}
	pCRP is a conference management app that runs on Google App Engine.
	pCRP is inspired by Eddie Kohler's HotCRP, \cite{kohler} but easier
	to set up and manage because there's no need to set up your own Linux
	server, PHP runtime, and MySQL server.
\section{Setup}
	You need Python 2.7.x (I recommend CPython) and the Google App Engine
	SDK for Python \cite{gaesdk} installed on your local machine. You also
	need pip, unless you prefer to install the dependencies manually. pip
	is included with Python 2.7.9; you'll probably have to install it
	yourself if you have an older version.

\bibliographystyle{plain}
\bibliography{manual}
\end{document}
