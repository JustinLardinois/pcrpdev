% Copyright 2014–2015 Justin Lardinois
%
% Licensed under the Apache License, Version 2.0 (the "License");
% you may not use this file except in compliance with the License.
% You may obtain a copy of the License at
%
%   http://www.apache.org/licenses/LICENSE-2.0
%
% Unless required by applicable law or agreed to in writing, software
% distributed under the License is distributed on an "AS IS" BASIS,
% WITHOUT WARRANTIES OR CONDITIONS OF ANY KIND, either express or implied.
% See the License for the specific language governing permissions and
% limitations under the License.
%%%%%%%%%%%%%%%%%%%%%%%%%%%%%%%%%%%%%%%%%%%%%%%%%%%%%%%%%%%%%%%%%%%%%%%%%%%%%%%
% docs/design.tex - a component of pCRP
% pCRP design document

\documentclass[12pt]{article}
\usepackage{fullpage}
\usepackage{hyperref}
\parindent 0pt

\title{pCRP 1.0.0 Design Document}
\author{Justin Lardinois}
\date{}

\begin{document}
\maketitle

\section{Architecture and Frameworks}
	Google App Engine does not support versions of Python newer than 2.7, so
	pCRP was written in Python 2.7. I did my best to avoid using constructs
	that aren't forwards compatible with Python 3. However, the Web Server
	Gateway Interface (WSGI) specification is not exactly compatible with
	Python 3 anyways \cite{python3}, so it's likely that my code won't be
	the main problem if there's ever an attempt to port pCRP to Python 3.
	\\\\
	At Professor Miller's suggestion, I built pCRP on the Flask
	microframework. Though Google App Engine will work with almost any
	framework that uses WSGI, only webapp and webapp2 are built in, and
	the official documentation and tutorials, and most other published
	materials about App Engine only cover those frameworks.
	\\\\
	Though Flask has been around for almost five years, it hasn't quite
	caught on in the App Engine realm. I was able to find exactly one
	tutorial that was specifically about the combination of App Engine
	and Flask \cite{souza}, and it was barely helpful. The Flask API
	documentation \cite{flask} is also less than stellar. I found my
	way through Flask through a combination of reading Stack Overflow
	and trial-and-error.
	\\\\
	Flask uses Jinja2 by default for templating, so I used Jinja2. All
	other dependencies with the exception of validate{\_}email are required
	by Flask. I had to incorporate validate{\_}email because App Engine's
	built in \texttt{is{\_}email{\_}valid} function returns \texttt{True} as
	long as the passed string is not \texttt{None} \cite{email}.
	\\\\
	As the older DB API is deprecated, I chose to use the NDB API for
	accessing the App Engine Datastore. NDB provides a few set data types
	for model properties, and uses repeated properties to represent lists.
	For other Python built in data structures, like sets and dicts,
	serialization is necessary. NDB supports both JSON and pickle, but
	since JSON does not play nice with sets and dicts, I use pickle.

\section{Security}
	One of the benefits of using App Engine and Flask is that the hardest
	security problems were solved for me. Google's Users API does a
	wonderful job of abstracting authentication away from the app, and
	Flask sets up the Jinja2 environment so that all data echoed into
	templates with .html extentions is HTML escaped.
	\\\\
	The Flask paradigm of using view functions to handle requests made
	it possible to verify many permissions with decorators. I defined
	\texttt{@login\_required}, \texttt{@registration\_required},
	\texttt{@admin\_only}, \texttt{@program\_committee\_only}, and
	\texttt{@pc\_chair\_only} for this purpose. This approach also
	has the added benefit of self-documentation, as one doesn't have
	to read the body of the view function to know who is allowed to
	access the page. These decorators are, of course, not the only
	verification used in the app; the views for editing, uploading, and
	viewing papers make sure that the user is the author of the given paper
	(or alternatively, in the case of viewing, a reviewer for the paper).
	\\\\
	pCRP only supports the upload of papers in PDF format, so it is of
	course necessary to verify that uploaded files are indeed PDFs. There
	are a few PDF processing libraries out there, but I felt that that was
	overkill. Instead, the app rejects uploaded files that do not begin with
	the \texttt{\%PDF} magic number, and serves the files back with a MIME
	type of \texttt{application/pdf}. The goal is that even if a user does
	manage to upload a non-PDF, viewers will get an error message instead of
	the file.

\section{Structure}
	The application design is best described as paper-centric. The majority
	of operations that a user can perform directly relate to papers; the
	rest of the operations impact which paper operations one can perform on
	which papers. The \texttt{Paper} model contains all information relevant
	to the paper: each program committee member's preference for that paper,
	the list of assigned reviewers, and the reviews themselves. Though the
	app does allow multiple authors to be listed for a paper, the user that
	created a paper is considered its ``true'' author; the other author
	fields are not associated with user accounts and are essentially just
	metadata.
	\\\\
	pCRP considers conflicts of interest to be unilateral, in the sense that
	it is only necessary for one person in a pair to indicate a conflict for
	there to be a bidirectional conflict. Conflicts are stored as a set of
	pairs of users, so for two users $A$ and $B$, there is a conflict if
	at least one of $(A,B)$ or $(B,A)$ is in the set.
	
\bibliographystyle{plain}
\bibliography{design}
\end{document}
