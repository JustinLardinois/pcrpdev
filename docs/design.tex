% Copyright 2014–2015 Justin Lardinois
%
% Licensed under the Apache License, Version 2.0 (the "License");
% you may not use this file except in compliance with the License.
% You may obtain a copy of the License at
%
%   http://www.apache.org/licenses/LICENSE-2.0
%
% Unless required by applicable law or agreed to in writing, software
% distributed under the License is distributed on an "AS IS" BASIS,
% WITHOUT WARRANTIES OR CONDITIONS OF ANY KIND, either express or implied.
% See the License for the specific language governing permissions and
% limitations under the License.
%%%%%%%%%%%%%%%%%%%%%%%%%%%%%%%%%%%%%%%%%%%%%%%%%%%%%%%%%%%%%%%%%%%%%%%%%%%%%%%
% docs/design.tex - a component of pCRP
% pCRP design document

\documentclass[12pt]{article}
\usepackage{fullpage}
\usepackage{hyperref}
\parindent 0pt

\title{pCRP 1.0.0 Design Document}
\author{Justin Lardinois}
\date{}

\begin{document}
\maketitle

\section{Architecture and Frameworks}
	Google App Engine does not support versions of Python newer than 2.7, so
	pCRP was written in Python 2.7. I did my best to avoid using constructs
	that aren't forwards compatible with Python 3. However, the Web Server
	Gateway Interface (WSGI) specification is not exactly compatible with
	Python 3 anyways \cite{python3}, so it's likely that my code won't be
	the main problem if there's ever an attempt to port pCRP to Python 3.
	\\\\
	At Professor Miller's suggestion, I built pCRP on the Flask
	microframework. Though Google App Engine will work with almost any
	framework that uses WSGI, only webapp and webapp2 are built in, and
	the official documentation and tutorials, and most other published
	materials about App Engine only cover those frameworks.
	\\\\
	Though Flask has been around for almost five years, it hasn't quite
	caught on in the App Engine realm. I was able to find exactly one
	tutorial that was specifically about the combination of App Engine
	and Flask \cite{souza}, and it was barely helpful. The Flask API
	documentation \cite{flask} is also less than stellar. I found my
	way through Flask through a combination of reading Stack Overflow
	and trial-and-error.
	\\\\
	Flask uses Jinja2 by default for templating, so I used Jinja2. All
	other dependencies with the exception of validate{\_}email are required
	by Flask. I had to incorporate validate{\_}email because App Engine's
	built in \texttt{is{\_}email{\_}valid} function returns \texttt{True} as
	long as the passed string is not \texttt{None} \cite{email}.

\bibliographystyle{plain}
\bibliography{design}
\end{document}
